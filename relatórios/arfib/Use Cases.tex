\documentclass{report}
\usepackage{graphicx}
\usepackage{enumitem}
\usepackage{multicol}
\usepackage[margin=2cm]{geometry}
\usepackage{hyperref}

\title{\Huge \textbf{Use Cases}\\ \huge Arfib}
\author{
    Henrique Romão \\ up202108067@up.pt
    \and
    Mariana Bessa \\ up202107946@up.pt
}

\setlength{\parindent}{0px}
\setlength{\parskip}{1em}

\begin{document}

\maketitle

\begin{abstract}
    Atrial fibrillation is one of the most prevalent cardiovascular conditions worldwide.
     To address the management of this condition, an application—Arfib—will be developed to support the monitoring and management of atrial fibrillation patients. 
     A key feature of the app is the ability for patients to upload real-time ECG measurements via the VitalJacket device. 
     Additionally, doctors have access to their patients' data, including ECG measurements, logged symptoms, and medication adherence, allowing for better-informed clinical decisions and improved patient care.
    
    \end{abstract}
%[comment] [bessa] faltam referencias aqui talvez
    \tableofcontents

\chapter{Requirements}
An app is being developed for atrial fibrilation monitoring.
This app is used by both patients monitoring this condition and respective doctors. 
This way, the app should allow for both patient and doctor separate interfaces.

A main part of this app is the ability to measure ECG signals, and display them in real-time while being measured.
A patient has to have the ability to make new measurements as well as consult old ones.
Each measurement should be analyzed, storing that information with the measurement result.

\textit{[Patient Symptoms]}
%[COMMENT] [BESSA] symptoms???

Patients can also track their medication in the app.
They should be able to log the medication that they need to take in each day, and see in which days they need to take which medications.
For each medication certain information needs to be available to see and edit, if needed, and patients can add new medication filling out the required information.
Patients should be able to see all medications in a simple view.

There is a blog inside the app, in which authorized personel -doctors- can publish different posts.
There should be an interface in which patients can see the blog posts and read each one in a dedicated page.

\textit{[Doctor Interface]}


Doctors have access to comprehensive information about all their patients through the application. 
They can view a list of their patients and, by selecting a specific patient, gain access to detailed records, including measurements, logged symptoms, and logged medications. 
Additionally, doctors have the capability to suggest medications directly through the application. 
This feature is particularly beneficial in cases where a prescribed medication has not yet been added by the patient.

To further support patient engagement and education, doctors can also contribute to the blog by publishing posts on relevant topics. 
They are able to create new posts as well as edit existing ones, ensuring their patients remain informed about key updates and health-related information.



\newpage
\chapter{Use Case Diagram}

\section{Actors}
Considering the system requirements, it was possible to identify the actors represented in Figure \ref{fig:Actors}.

\subsection{Diagram}
\begin{figure}[ht]
    \centering
    \includegraphics[width=0.5\linewidth]{Actors.pdf}
    \caption{Actors.}
    \label{fig:Actors}
\end{figure}

\subsection{Patient}
\subsection{Doctor}
\subsection{Nurse}

\clearpage
\section{Use Cases}
Considering the actors and system requirements, the following diagram represents the use cases for this system.

\begin{figure}[hb]
    \centering
    \includegraphics[width=\linewidth]{General Use Case Diagram.pdf}
    \caption{Use Case Diagram.}
    \label{fig:Use Case}
\end{figure}

\chapter{Use Case Description}
\vspace{-3em}
\section{Measurement}
\subsection{Log Measurement}
\vspace{-1em}
\paragraph{Brief Description}
This use case allows a patient to perform an ECG measurement using a device of their choosing. 
The patient must select a valid device, being abble to add new devices.
The measurement is shown in real time and can be stopped during the process.

\vspace{-1em}
\subsubsection{Flow of Events}
\begin{figure}[ht]
    \centering
    \includegraphics[width=0.7\linewidth]{Log Measurement.pdf}
    \caption{Log Measurement Activity Diagram.}
    \label{fig:Log Measure}
\end{figure}

\clearpage
\subsection{See Measurement}

\paragraph{Brief Description}
This use case allows a patient see its own previous measurements, or allows a doctor to see the previous measurements of their patients.
Both have access to a list of the measurments, allowing them to choose a specific one to see in more detail.

\begin{multicols}{2}
    \paragraph{Basic Flow}
    \begin{enumerate}
        \item Select patient's measurements page.
        \item Check access.
        \item Display previous measurements.
        \item Choose measurement of interest.
        \item Display full information about measurement of interest.
    \end{enumerate}
    \columnbreak
    
    \paragraph{Alternative Flows}
    \begin{enumerate}[label=A\arabic*.]
        \item Doctor or patient does not have access.
        \item Show error message.
        \item Quit.
        \item See measurement closed.
    \end{enumerate}
\end{multicols}

\vspace{1em}
\section{Medications}
\subsection{Log Medication}
\paragraph{Brief Description}
This use case allows a patient to affirme if they have taken their medication, and store that information.

\begin{multicols}{2}
    \paragraph{Basic Flow}
    \begin{enumerate}
        \item Select medication to log.
        \item Confirm medication has been taken.
        \item Change medication status.
    \end{enumerate}
    \columnbreak

    \paragraph{Alternative Flows}
    \begin{enumerate}[label=A\arabic*.]
        \item Select medication to remove log.
        \item Confirm mistake.
        \item Change status of the mistakenly logged medication.
    \end{enumerate}
\end{multicols}

\vspace{1em}
\subsection{See Medications}
\paragraph{Brief Description}
This use case allows a patient see its medications and respective logs, or allows a doctor to see the medications and logs of their patients.
Both have access to the medication's logs and to the medications' information.

\begin{multicols}{2}
    \paragraph{Basic Flow}
    \begin{enumerate}
        \item Select patient's medications page.
        \item Check access.
        \item Display medication logs.
        \item Choose medication of interest.
        \item Display information about medication of interest.
    \end{enumerate}
    \columnbreak

    \paragraph{Alternative Flows}
    \begin{enumerate}[label=A\arabic*.]
        \item Doctor or patient does not have access.
        \item Show error message.
        \item Quit.
        \item See medication closed.
    \end{enumerate}
\end{multicols}

\vspace{1em}
\subsection{Edit Medication}
\paragraph{Brief Description}
Allows the patient to change information regarding their medication, such as intake quantity, frequency and select an end date, from which the medication will not need to be taken.
Also, the use case allows adding new medications.

\begin{multicols}{2}
    \paragraph{Basic Flow}
    \begin{enumerate}
        \item Select medication.
        \item Enter edit mode.
        \item Choose parameter to change.
        \item Exit edit mode.
    \end{enumerate}
    \columnbreak

    \paragraph{Alternative Flows}
    \begin{enumerate}[label=A\arabic*.]
        \item Select new medication.
        \item Fill the required information.
        \item Add new medication.
    \end{enumerate}
\end{multicols}

\vspace{1em}
\subsection{Suggest Medication}
\paragraph{Brief Description}
This use case allows a doctor to suggest a new medication for a given patient.

\paragraph{Basic Flow}
\begin{enumerate}
    \item Select patient.
    \item Select medication.
    \item Fill the required information.
    \item Check information.
    \item Send to patient.
    \item Alert patient of suggestion.
\end{enumerate}
% [COMMENT][HENRIQUE] aqui não estou a ver nenhum alternative flow
\vspace{1em}
\subsection{See Symptoms}
\paragraph{Brief Description}
Patients can review their previously logged symptoms in a manner similar to how doctors access a patient’s symptom history. 
This feature offers two views: a calendar view, which provides an overview of all symptoms logged on each day, and a symptom-specific view, which focuses on a single symptom, displaying a calendar with its occurrences and corresponding intensity levels.

\begin{multicols}{2}
    \paragraph{Basic Flow}
    \begin{enumerate}
        \item Select patient's symptoms page.
        \item Select desired view (calendar view or symptom-specific view).
        \item Display symptoms logs.
    \end{enumerate}
    \columnbreak

    \paragraph{Alternative Flows}
    \begin{enumerate}[label=A\arabic*.]
        \item Switch to a different view (e.g., from calendar view to symptom-specific view).
        \item Select a different symptom while in symptom-specific view.
        \item Select a different day while in calendar view.
\end{enumerate}
\end{multicols}

%[COMNENT] [BESSA] nao sei se esta mt bem o alternate

\vspace{1em}
\section{Blog}
\subsection{See Post}
\paragraph{Brief Description}
This use case allows both doctors and patients to select a post from a catalog and view its detailed information.

\begin{multicols}{2}
    \paragraph{Basic Flow}
    \begin{enumerate}
        \item Open blog page.
        \item Choose the desired post.
        \item Display the post in detail.
    \end{enumerate}
    \columnbreak

    \paragraph{Alternative Flows}
    \begin{enumerate}[label=A\arabic*.]
        \item Select a different post.
        \item Exit the blog page.
        \item Error message if the post does not load.
    \end{enumerate}
\end{multicols}

\vspace{1em}
\subsection{Publish Post}
\paragraph{Brief Description}
This use case allows doctors to publish a post on the application's blog, concerning topics relevant to atrial fibrilation.

\begin{multicols}{2}
    \paragraph{Basic Flow}
    \begin{enumerate}
        \item Select blog's page.
        \item Enter edit mode.
        \item ompose the post by writing the title and content.
        \item Confirm the publication.
        \item Exit edit mode after publishing.
        \item Notify the doctor's patients of a new post.
    \end{enumerate}
    \columnbreak

    \paragraph{Alternative Flows}
    \begin{enumerate}[label=A\arabic*.]
        \item Add images to the post.
        \item Cancel the publishment.
        \item Error message during upload.
    \end{enumerate}
\end{multicols}

\end{document}